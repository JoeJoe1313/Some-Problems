\documentclass{article}
\usepackage[T1,T2A]{fontenc}
\usepackage[utf8]{inputenc}
\usepackage[bulgarian]{babel}
\usepackage[document]{ragged2e}
\usepackage{amssymb}
\usepackage{amsmath}

\title{Държавен изпит - 05 август 2020 г.}
\author{Йоана Левчева }

\begin{document}

\maketitle

\section*{Задача 2}

\justify

Нека $e = \{e_1, e_2, e_3, e_4\}$ е ортонормиран базис на евклидовото пространство $\mathbb{R}^4$ и $\varphi\in$ $Hom(\mathbb{R}^4)$ е линеен оператор, действащ по правилото:

\begin{center}
    $\varphi(\xi_1 e_1 + \xi_2 e_2 + \xi_3 e_3 + \xi_4 e_4) = $ \\
    $(\alpha\xi_1 + 2\xi_2-2\xi_3 + 3\xi_4)e_1 + $ \\
    $(2\xi_1 + \alpha\xi_2 + 3\xi_3 - 2\xi_4)e_2 + $ \\
    $(-2\xi_1 + 3\xi_2 + \alpha\xi_3 + 2\xi_4)e_3 + $ \\
    $(3\xi_1 - 2\xi_2 + 2\xi_3 + \alpha\xi_4)e_4, $
\end{center}

\justify
където $\alpha \geq 0$ е реален параметър.

\justify
$a)$ Да се намери матрицата $A$ на оператора $\varphi$ в този базис. \\
$b)$ tbd \\
$c)$ tbd \\
$d)$ tbd \\
$e)$ tbd

\justify
Решение:

\justify
Първо за $(\xi_1, \xi_2, \xi_3, \xi_4) = (1, 0, 0, 0)$ изчисляваме $\varphi(e_1) = \alpha e_1 + 2e_2 - 2e_3 + 3e_4$, за $(\xi_1, \xi_2, \xi_3, \xi_4) = (0, 1, 0, 0)$ изчисляваме $\varphi(e_2) = 2e_1 + \alpha e_2 + 3e_3 - 2e_4$, за $(\xi_1, \xi_2, \xi_3, \xi_4) = (0, 0, 1, 0)$ изчисляваме $\varphi(e_3) = -2e_1 + 3e_2 + \alpha e_3 + 2e_4$, за $(\xi_1, \xi_2, \xi_3, \xi_4) = (0, 0, 0, 1)$ изчисляваме $\varphi(e_4) = 3e_1 - 2e_2 + 2e_3 + \alpha e_4$. Така за матрицата $A$ получаваме

\begin{center}
    A = $\begin{pmatrix}
         \alpha & 2 & -2 & 3 \\
         2 & \alpha & 3 & -2 \\
         -2 & 3 & \alpha & 2 \\
         3 & -2 & 2 & \alpha
         \end{pmatrix}$
\end{center}

\justify
Забелязваме, че сумата от елементите на всеки ред на матрицата $A$ e равна на едно и също число, а именно $r := \alpha + 3$. 

\justify
\underline{Свойство:} Ако една матрица $A_{4\times4}$ има редове с елементи, даващи една и съща сума $r$, то $r$ е собствено число на матрицата $A$, a $(1,1,1,1)$ е собствен вектор, съответстващ на това собствено число.

\justify
Доказателство:

\justify
Нека е дадена матрицата $A_{4\times4}$, която има редове с елементи, даващи една и съща сума $r$. Имаме, че

\begin{center}
    $A\begin{pmatrix}
         1 \\
         1 \\
         1 \\
         1
         \end{pmatrix} = \begin{pmatrix}
         r \\
         r \\
         r \\
         r
         \end{pmatrix},$
\end{center}

\justify
което е еквивалентно на 

\begin{center}
    $A\begin{pmatrix}
         1 \\
         1 \\
         1 \\
         1
         \end{pmatrix} = r\begin{pmatrix}
         1 \\
         1 \\
         1 \\
         1
         \end{pmatrix}.$
\end{center}

\justify
Очевидно, $r$ е собствено число, а $(1,1,1,1)$ е собствен вектор, съответстващ на това собствено число.

\justify
Следователно за нашата матрица $A$ имаме, че едно собствено число е $\lambda_1 = \alpha + 3$, а $(1,1,1,1)$ е собствен вектор, съответстващ на това собствено число. 

\justify
По условие $e = \{e_1, e_2, e_3, e_4\}$ е ортонормиран базис на евклидовото пространство. $А$ е матрицата на $\varphi$ в този базис. Но матрицата $A$ е симетрична, откъдето следва, че операторът $\varphi$ е симетричен. Да отбележим още, че всеки два собствени вектора, съответстващи на различни собствени стойности на симетричен оператор, са ортогонални помежду си. Следователно, достатъчно е да намерим още два ортогонални вектора на $(1,1,1,1)^{T}$, такива че също да са ортогонални помежду си. Два такива вектора например са $(1,-1,1,-1)^{T}$ и $(1,1,-1,-1)^{T}$ и чрез тях ще намерим още две собствени стойности. (Четвъртата собствена стойност ще намерим, използвайки следата на матрицата.)

\justify
За да намерим $\lambda_2$, решаваме системата

\begin{center}
    $A\begin{pmatrix}
         1 \\
         -1 \\
         1 \\
         -1
         \end{pmatrix} = \lambda_2 \begin{pmatrix}
         1 \\
         -1 \\
         1 \\
         -1
         \end{pmatrix}$
\end{center}

\justify
и получаваме $\lambda_2 = \alpha - 7$.

\justify
За да намерим $\lambda_3$, решаваме системата

\begin{center}
    $A\begin{pmatrix}
         1 \\
         1 \\
         -1 \\
         -1
         \end{pmatrix} = \lambda_3 \begin{pmatrix}
         1 \\
         1 \\
         -1 \\
         -1
         \end{pmatrix}$
\end{center}

\justify
и получаваме $\lambda_3 = \alpha + 1$.

\justify
За $\lambda_4$ ще използваме факта (might explain it later), че $tr(A) = \lambda_1 + \lambda_2 + \lambda_3 + \lambda_4$, откъдето получаваме $\lambda_4 = \alpha + 3$.

\justify
Сега по условие $x_1 = x_2$, т.е. $x_1 = x_ 2 = \alpha + 3$, $x_3 < x_4$, т.е. $x_3 = \alpha - 7, x_4 = \alpha + 1$ и $x_1 + x_2 + x_3 = 2x_4$, което е еквивалентно на $\alpha + 3 + \alpha + 3 + \alpha - 7 = 2\alpha + 2$. От тук получаваме, че $\alpha = 3$ и матрицата $A$ има вида

\begin{center}
    A = $\begin{pmatrix}
         3 & 2 & -2 & 3 \\
         2 & 3 & 3 & -2 \\
         -2 & 3 & 3 & 2 \\
         3 & -2 & 2 & 3
         \end{pmatrix}.$
\end{center}

\justify
Трябва да намерим собствения вектор, съответстващ на $\lambda_4 = 6$, който да е ортогонален с останалите собствени вектори. За целта трябва да решим системата 

\begin{center}
    $Av_4 = \lambda_4 v_4$, \\
    $(A - E\lambda_4)v_4 = 0,$ \\
    $\begin{pmatrix}
         -3 & 2 & -2 & 3 \\
         2 & -3 & 3 & -2 \\
         -2 & 3 & -3 & 2 \\
         3 & -2 & 2 & -3
         \end{pmatrix} v_4 = 0$
\end{center}

\justify
Let $A$ be a matrix.  It has a Jordan Canonical Form, i.e. there is matrix $P$ such that $PAP^{-1}$ is in Jordan form.  Among other things, Jordan form is upper triangular, hence it has its eigenvalues on its diagonal.  It is therefore clear for a matrix in Jordan form that its trace equals the sum of its eigenvalues.  All that remains is to prove that if $B,C$ are similar then they have the same eigenvalues.

\end{document}
